% !TEX TS-program = xelatex
% !TEX encoding = UTF-8

% This is a simple template for a XeLaTeX document using the "article" class,
% with the fontspec package to easily select fonts.

\documentclass{article} % use larger type; default would be 10pt

\usepackage{bidi}
\usepackage{fontspec} % Font selection for XeLaTeX; see fontspec.pdf for documentation
\defaultfontfeatures{Mapping=tex-text} % to support TeX conventions like ``---''
\usepackage{xunicode} % Unicode support for LaTeX character names (accents, European chars, etc)
\usepackage{xltxtra} % Extra customizations for XeLaTeX

\setmainfont[Scale=1.4]{Minion Pro} % set the main body font (\textrm), assumes Charis SIL is installed
\setsansfont{Myriad Pro}  % calibri
\setmonofont{Courier New} % consolas, calibri
\newfontfamily\arabicfont[Script=Arabic,Scale=1.4]{Adobe Arabic}
\usepackage{arabxetex}

% other LaTeX packages.....
\usepackage{geometry} % See geometry.pdf to learn the layout options. There are lots.
\geometry{a4paper} % or letterpaper (US) or a5paper or....
%\usepackage[parfill]{parskip} % Activate to begin paragraphs with an empty line rather than an indent

\usepackage{graphicx} % support the \includegraphics command and options

\title{Definitions}
\date{} % Activate to display a given date or no date (if empty),
         % otherwise the current date is printed 

\begin{document}
\maketitle

\begin{itemize}
\item[\textbf{character}] A member of a set of elements used for the organisation, control, or 
representation of data. (ISO/IEC 10646-1: 
2003)

E.g \texttt{ARABIC LETTER ALEF},\texttt{ARABIC LETTER BEH}  (ISO/IEC 10646-1: 2003)

\item[\textbf{glyph}] A recognizable abstract 
graphic symbol which is independent of any 
specific design. (ISO/IEC 9541-1: 1991) 

e.g the glyph for the characters \texttt{ARABIC LETTER ALEF}, \texttt{ARABIC LETTER BEH}, \texttt{ARABIC LETTER TEH} are ; 
{\fontspec[Script=Arabic, Scale=2]{Adobe Arabic} ا , ب , ت 
}

A character may have many different glyphs. e.g {\fontspec[Script=Arabic,Scale=2]{Adobe Arabic} ب}, {\fontspec[Script=Arabic,Scale=2]{Arabic Typesetting} ب}, {\fontspec[Script=Arabic,Scale=2]{Times New Roman} ب}, {\fontspec[Script=Arabic,Scale=2]{Arial} ب}


\item[\textbf{font}] A collection of glyph images 
having the same basic design, e.g. \textsf{Arabic Typesetting}, \textsf{Adobe Arabic}. (ISO/IEC 9541-1: 1991)

font type \texttt{Adobe Arabic} \hskip 1cm {\fontspec[Script=Arabic,Scale=2]{Adobe Arabic} ساي سوکا جاوي}

font type \texttt{Arabic Typesetting} \hskip 1cm  {\fontspec[Script=Arabic,Scale=2]{Arabic Typesetting} ساي سوکا جاوي}

font type \texttt{Arial} \hskip 2cm  {\fontspec[Script=Arabic,Scale=2]{Arial} ساي سوکا جاوي}

\item[\textbf{Coded character}] A character together with its coded representation. E.g The character \texttt{ARABIC LETTER ALEF} is coded \texttt{U+0627}. Each character has a unique code. (ISO/IEC 10646-1: 
2003)
\end{itemize}

In information technology,  characters  are 
abstract information elements in the domain 
of coding for data representation and, in 
particular, data interchange. 

Coded character set standards assign numeric values, 
character names, and representative (sample) images to each character contained in 
a coded character set. Typically a character 
is given a name, which also serves to differentiate it from the other characters of the 
coded character set. 

The precise semantics 
and appearance of the information elements 
in any given implementation are not defined 
by those standards for coded character 
sets. This apparent lack of definition is not 
considered to be a defect in the standards. 
Recognizing that the information may be 
acted upon (deciphered, sorted, transformed, formatted, archived, presented, 
etc.) by many different application proc-
esses during its lifetime, standards for 
coded character sets are defined as a basis 
for information interchange.

 Characters and  glyphs are closely related, 
with many attributes in common and yet 
with distinctions that make it essential that 
they be managed in information processing 
as separate entities. 

The ISO/IEC 10646 
standard recognizes the distinction between characters and their visual representation 
by defining the term,  graphic symbol. The 
graphic symbol of SC 2 standards and the 
glyph image of SC 18 standards represent 
equivalent concepts. 

However, glyph and its 
associated ISO/IEC 9541 terminology are 
preferred when referring to presentation and 
presentation processing.  
%\section{}

%\subsection{}



\end{document}
